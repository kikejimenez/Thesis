\chapter{Notación y Convenciones}
\label{ApenA}
\lhead{Apéndice A. \emph{Notación y convenciones}}
Durante todo el texto, hemos usado la notación y las convenciones de la referencia~\cite{Weinberg:1995mt}, las cuales escribimos en este apéndice. 

Representamos los índices de Dirac por las  primeras letras del alfabeto griego, $ \alpha,\alpha',\beta,\beta' $, etcétera. Los índices de Lorentz están siendo representados por las últimas letras del mismo alfabeto, $ \mu,\nu,\mu',\nu' $,  etcétera. Tomamos la métrica de Lorentz como 
\begin{IEEEeqnarray}{rl}
               \eta_{\mu\nu}  \, = \, \text{diag}\begin{pmatrix}
1 & 1 & 1 & -1
\end{pmatrix} \ ,
      \label{Ap-A-01}
  \end{IEEEeqnarray}  
  siguiendo la costumbre de denotar la última componente con el número cero. La representación de Dirac [la representación $ \left( \frac{1}{2},0 \right)\oplus \left( 0,\frac{1}{2} \right)$] 
\begin{IEEEeqnarray}{rl}
             D\left[ \Lambda\right]   \, = \, \exp\left[\tfrac{i}{2} w_{\mu\nu}\mathcal{J}^{\mu \nu} \right],
    \label{Ap-A-02}
\end{IEEEeqnarray}
es generada por las matrices 
\begin{IEEEeqnarray}{rl}
             \quad     \mathcal{J}^{\mu \nu} \, = \, \tfrac{-i}{4}[\gamma^{\mu}, \gamma^{\nu}] \ .
    \label{Ap-A-03}
\end{IEEEeqnarray}
Aquí, los símbolos $ \gamma^{\mu} $ son las matrices de Dirac que satisfacen las relaciones de anticonmutación con signo positivo,
\begin{IEEEeqnarray}{rl}
             \left\lbrace  \gamma^{\mu} ,  \gamma^{\nu}\right\rbrace   \, = \, 2\eta^{\mu\nu} 
    \label{Ap-A-04}
\end{IEEEeqnarray}
 Nos apegamos al siguiente conjunto particular para las matrices $ \gamma^{\mu} $:
\begin{equation}
         \gamma^{0}\, = \,-i\begin{pmatrix}
0 & I \\ 
I & 0
\end{pmatrix} , \quad {\gamma}_{i}\, = \,-i  \begin{pmatrix}
0 & \sigma_{i} \\ 
-\sigma_{i} & 0
\end{pmatrix}\ .
         \label{Ap-A-05}
	\end{equation}
donde las matrices $ \sigma_{i} $, son las matrices de Pauli,
\begin{IEEEeqnarray}{rl}
            \sigma_{1}  \, = \, \begin{pmatrix}
0 & 1 \\ 
1 & 0
\end{pmatrix} , \quad  \sigma_{2}  \, = \, \begin{pmatrix}
0 & i \\ 
-i & 0
\end{pmatrix} , \quad  \sigma_{3}  \, = \, \begin{pmatrix}
1 & 0 \\ 
0 & -1
\end{pmatrix} ,
    \label{Ap-A-06}
\end{IEEEeqnarray}
La matriz $ \gamma_{5} \equiv i  \gamma^{0} \gamma^{1} \gamma^{2}\gamma^{3}  $, viene dada por 
\begin{IEEEeqnarray}{rl}
            \gamma_{5}  \, = \, \begin{pmatrix}
I & 0 \\ 
0 & -I
\end{pmatrix} 
    \label{Ap-A-07}
\end{IEEEeqnarray}
Introducimos
\begin{IEEEeqnarray}{rl}
         \epsilon  \, = \, \begin{pmatrix}
 e & 0 \\ 
 0 & e
 \end{pmatrix} , \quad e  \, = \, \begin{pmatrix}
 0 & 1 \\ 
 -1 & 0
 \end{pmatrix} ,
    \label{Ap-A-08}
\end{IEEEeqnarray}
y
\begin{equation}
      \beta\, = \,\begin{pmatrix}
0 & I \\ 
I & 0
\end{pmatrix}  \ .
         \label{Ap-A-09}
	\end{equation}
Estas matrices nos ayudan a convertir  las representación transpuesta-inversa y transpuesta-conjugada de la representación de Dirac, ya que satisfacen las relaciones
 \begin{IEEEeqnarray}{rl}
               \beta \gamma^{\mu}   \, = \,  -\gamma^{\mu\dagger}\beta, \quad   \epsilon\gamma_{5} \gamma^{\mu} \, = \, -\gamma^{\mu \intercal}\epsilon\gamma_{5}\ .
     \label{Ap-A-10}
 \end{IEEEeqnarray}	
 y por lo tanto 
\begin{IEEEeqnarray}{rl}
             D(\Lambda^{-1})^{\dagger}  \, = \, \beta D(\Lambda)\beta ^{-1}, \quad  D(\Lambda^{-1})^{\intercal}  \, = \,  \epsilon\gamma_{5} D(\Lambda)\left( \epsilon\gamma_{5}\right)  ^{-1}\ .
    \label{Ap-A-11}
\end{IEEEeqnarray}
Esto nos demuestra  que las representaciones $ D(\Lambda)^{*} $  y $ D(\Lambda)^{\intercal}  $ son representaciones equivalentes  a $ D(\Lambda) $ (en el sentido de la teoría de representaciones).