\chapter{Conclusiones}
\label{Chap:Conclusiones}
\epigraph{``\textit{My brain is hanging upside down\\
             I need something to slow me down}"}{The Ramones}
\lhead{Cap\'itulo 8. \emph{Conclusiones}}


En esta tesis hemos resuelto el problema de encontrar funciones de onda en el superespacio, para superpartículas  masivas de cualquier superespín. El sustento de este trabajo  y el desarrollo del mismo, descansan en el artículo~\cite{Jimenez:2014gfa}.\\

Ha sido posible desarrollar la teoría cuántica de los campos en el superespacio, usando puramente el formalismo de operadores y sin recurrir a ninguna regla de cuantización. Nuestro punto de partida ha sido el de promover  el enfoque de Weinberg a la teoría del campo, del espacio al superespacio.   Hemos logrado:
\begin{itemize}
\item[-] Construir bases para el superespacio de Hilbert. A diferencia del caso en el espacio, los superestados encontrados normalizan con el mapeo exponencial fermiónico. 
\item[-] Construir superestados completamente covariantes bajo transformaciones arbitrarias del grupo de super Poincaré. Hemos demostrado que estos supermultipletes son equivalentes a los estados encontrados usando los métodos usuales en componentes.
\item[-] Establecer  que es lo que entendemos por una supermatriz $ S $ covariante. 
\item[-] Obtener superamplitudes perturbativas en forma de  diagramas de super Feynman. Esto se ha hecho introduciendo interacciones que usan dos supercampos quirales $ \Phi_{+\ell} $ y  $ \Phi_{-\ell} $ y sus adjuntos, para cualquier representación del grupo de Lorentz.
\item[-] Dar fórmulas explícitas para  las transformaciones de los supercampos bajo transformaciones de paridad, carga, inversión temporal y simetrías $ \mathcal{R} $.
\item[-] Establecer una equivalencia explícita entre nuestro método y el formalismo canónico para el caso del supercampo quiral masivo (el afamado modelo de Wess-Zumino).
\item[-] Demostrar la conveniencia de nuestra formulación mediante el cálculo a orden más bajo  en teoría de perturbaciones de una superpartícula escalar con su respectiva antispartícula.
\end{itemize}
\begin{center}
\textbf{\textit{Perspectivas futuras}}
\end{center}
Somos de la opinión que nuestros descubrimientos pueden abrir varias puertas para seguir avanzando en el entendimiento de las teorías supersimétricas. Nos permitimos citar algunas líneas de investigación que creemos son asequibles en el corto y mediano plazo y que además son de interés para la comunidad que trabaja en supersimetría:
\begin{itemize}
\item[-] Las  generalización del formalismo para supersimetrías $ \mathcal{N} $-extendidas se puede realizar de manera directa, puesto que los estados generales de superpartícula en  el superespacio de supermomento $ \mathcal{N} $-extendido pueden ser definidos en términos de los estados  en  el superespacio de supermomento $ (\mathcal{N} -1)$-extendido. Esto es, los superestados presentados en la   secci\'on \eqref{chap:2-4} admiten una generalización recursiva.
\item[-] Nuestra propuesta podría encontrar aplicaciones más all\'a de la teor\'ias con superespín arbitrario, por ejemplo, extendiendo resultados en formulaciones de la teoría cuántica de los campos, basadas en el formalismo de operadores, del caso del  espacio al superespacio. La obtención de los superestados de muchas superpartículas, $ \left| \mathcal{A} \right\rangle $, que se transforman  de manera completamente covariante bajo transformaciones de super Poincar\'e, ha hecho posible que los elementos generales de matriz:
\begin{IEEEeqnarray}{rl}
             \left\langle  \mathcal{M}\left| \mathcal{O}(z_{_{1}},\dots, z_{_{n}})\right|\mathcal{N}\right\rangle \ ,
    \label{8--}
\end{IEEEeqnarray}
(para operadores  $ \mathcal{O} $  en el superespacio, formados con supercampos en el esquema de Heisenberg, evaluados en  $ (z_{_{1}},\dots, z_{_{n}}) $ y posiblemente ordenados temporalmente), sean expresados  como otros elementos de matriz evaluados en los puntos $ z_{_{1}}-z,\dots,  z_{_{n}}-z  $, con $ z $ arbitrario. Estas traslaciones son usadas en elementos de matriz intermediarios que se encuentran en trabajos basados en el formalismo de operadores, como por ejemplo:
\begin{itemize}
\item[-] Las representaciones espectrales~\cite{Kallen:1952zz,Lehmann:1954}. Hasta el momento, solamente resultados  obtenidos en el contexto de los métodos funcionales y limitados al caso del supercampo escalar son conocidos~\cite{Constantinescu:2003vn}).
\item[-] La expansión en productos de operadores (OPE)~\cite{Wilson:1969zs}.  Al igual que el caso de la representación espectral, solamente el caso escalar es conocido~\cite{Leroy:1986ve}).
\item[-] Simetrías globales rotas espontáneamente~\cite{Goldstone:1962es}.
\end{itemize}
\item[-] Escribir de una manera completamente covariante resultados que normalmente son presentados en componentes, tales  como las restricciones cinematicas en supergravedad~\cite{Grisaru:1976vm} y las amplitudes a nivel árbol en QCD que se obtienen usando amplitudes supersimétricas~\cite{Dixon:1996wi}.  
\item[-] Siguiendo la misma línea que se encuentra entre las formulaciones Lagrangianas y puramente de la matriz $ S $, podríamos extender los resultados de 
\begin{itemize}
\item[-] La formulación las reglas de Feynman para partículas sin masa~\cite{Weinberg:1964ev}, cuyo establecimiento debe ser directo, pero el comparativo con el límite de masa cero de  nuestros resultados sera útil. 
\item[-] Dimensiones extra~\cite{Weinberg:1984vb}.
\item[-] Teorías del campo  con invariancia de escala y conforme~\cite{Chan:1973iq,Weinberg:2012cd}.
\end{itemize}
\item[-] Extensiones para obtener superfunciones de onda  para el caso de teorías de norma. Vemos como  asequible investigar en las líneas de las referencias ~\cite{Weinberg:1965rz,Weinberg:1965nx,Grisaru:1976vm} (de las cuales  evidencia de nuevos teoremas suaves y nuevas relaciones de Ward han sido recientemente encontradas~\cite{Cachazo:2014fwa,He:2014laa}).
\item[-] Pertubativemente, la mayoria de las teorías supersimétricas rotas (mediante algún mecanismo) preservan el número de partículas de las teorías exactas. Entonces, el formalismo aquí presentado puede  en principio ser extendido para calcular amplitudes en teorías supersimétricas rotas. Esto podria realizarse extendiendo las super reglas de Feynman para incluir términos explícitos de rompimiento de supersimétrica que se originan como constantes de acoplamiento locales en las variables fermionicas (llamados usualmente campos externos espurios). Estas ideas han sido explotadas usando métodos funcionales, en teorías rotas espontáneamente, para encontrar relaciones explícitas entre contratérminos ultravioletas de los términos de rompimiento espontaneo y teorías supersimétricas puramente rígidas (simetría global)~\cite{Avdeev:1997vx,Kobayashi:1998jq}.
\end{itemize}