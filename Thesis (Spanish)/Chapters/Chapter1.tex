\chapter{Introducción}
\label{sec:Intro}


\epigraph{\textit{``I did not always know who was responsible for material presented here, and the mere absence of a citation should not be taken as a claim that the material presented here is original. But some of it is."}}{S. Weinberg~\cite{Weinberg:1995mt}}

\lhead{Cap\'itulo 1. \emph{Introducción}}

Existen dos conjuntos, ya con larga tradición, de fanáticos de la supersimetría. En un espectro están aquellos que su labor  es proponer modelos, los llamados  ``hacedores de modelos'' y fenomenólogos,   mientras están los otros que se concentran en el método, cuya tarea es proponer nuevos formalismos para la misma o una nueva teoría. Aunque supersimetría es un adulto maduro que ronda los   40 años edad, aún existen enigmas en ambas escuelas de trabajo.  Obviamente, de importancia ulterior para la  Física fundamental es  saber si supersimetría se realiza o no en la Naturaleza. Se espera  que el  Gran Colisionador de Hadrones en su segunda corrida, nos ayude a discernir esta cuestión. 

 El estudio de los modelos físicos pone en evidencia qué se necesita para poder hacer una descripción adecuada de la física en términos matemáticos y confronta los formalismos con la realidad. Mientras que la aplicación de un nuevo método matemático, que resulta consistente con las teorías físicas conocidas,  nos pone en una posición de entender mejor nuevas realidades. En este último aspecto,  es pertinente (o casi obligado) citar al Físico soviético L. D. Landau:
\textit{A method is more important than a discovery, since the right method will lead to new and even more important discoveries~\footnote{Ver \url{ https://en.wikiquote.org/wiki/Lev\_Landau.}}.}  Ambos enfoques son esencialmente necesarios para avanzar nuestro conocimiento científico.   


Independientemente de si tiene algo que ver con la naturaleza o no, supersimetría llegó para quedarse.  Esto por  el simple he hecho que durante todo su desarrollo nos ha ofrecido \emph{entendimiento}. Desde los teoremas de no-renormalización~\cite{Grisaru:1979wc} hasta la solución de Seiberg-Witten~\cite{Seiberg:1994rs}, las peculiaridades de la supersimetría han fascinado a propios y extraños. Al día de hoy,  la única apuesta segura de supersimetría es la de seguir ofreciéndonos compresión en varias facetas  de la física teórica, la física matemática y la  física en general. Supersimetría es un laboratorio teórico donde los experimentos son muy baratos~\footnote{Este sentencia puede considerarse como un estancia particular de la aseveración más general de V.I. Arnold,
\emph{``Mathematics is the part of physics where experiments are cheap.'' \url{http://pauli.uni-muenster.de/~munsteg/arnold.html}}}.

En esta tesis nos concentramos primordialmente en el método, teniendo como uno de los principales objetos de estudio,  el superespaciotiempo. No nos enfocamos en  las conexiones que existe entre el superespacio y la geometría (o supergeometría), de la cual muchas páginas han sido escritas~\footnote{El libro por excelencia de la supergeometría, es la Ref. \cite{dewitt1992supermanifolds}. Un tratado más reciente es la Ref. \cite{rogers2007supermanifolds}.}, sino de la relaciones entre las simetrías del superespaciotiempo y las reglas de la Mecánica Cuántica. 

Antes de expresar de manera explícita lo que se hecho en este trabajo, es pertinente  ver brevemente a la supersimetría  sumergidos en la  visión de la época en que nació.   El surgimiento de la supersimetría se da en un contexto donde la integral de caminos ya había sido popularizada por los trabajos de G. t'Hooft~\cite{'tHooft:1971rn} sobre la teoría electrodébil (en  contraste, tenemos que   la  cuantización canónica de los campos  está enraizada con los origines de la teoría cuántica misma~\cite{dirac2001lectures}).  Dentro de la integral de caminos, los denominados métodos de campo externo~\cite{weinberg1996quantum} probaron ser de gran utilidad  y generalidad (esto es, en este formalismo variamos corrientes externas que después apagamos y donde todos los efectos de los multilazos quedan codificados en la \emph{acción cuántica efectiva}). Es en este contexto, donde la supersimetría se formula (al menos en el Oeste) por primera vez~\cite{Wess:1973kz}. Así también,  desde que A. Salam y J. Stradhee introducen la idea el superespacio~\cite{Salam:1974jj}, la teoría del campo en el superespacio ha sido desarrollada usando  métodos funcionales~\cite{Salam:1974pp,Salam:1974yz,Grisaru:1976vm,Grisaru:1979wc}.  
 
  Aunque la generalidad del método de funcionales en teoría del campo lo hace muy conveniente (cuyo esplendor se puede apreciar en el formalismo de Batalin-Vilkovisky~\cite{Batalin:1981jr}), existen varios aspectos que se vuelven menos claros en este enfoque. En particular, con  los métodos funcionales obtenemos de manera directa las funciones de correlación de  $ n $ puntos de cualquier teoría del campo. Expresado en términos  de diagramas de Feynman, las funciones de correlación, representan la  suma de todos los diagramas con $ n $   patas externas \textit{fuera} de la capa de masa. Para obtener una amplitud de un proceso físico, necesitamos reemplazar las patas fuera de la capa de masa por las correspondientes patas \textit{en} la capa de masa. Como hemos dicho anteriormente, ya que en el superespacio sólo los métodos funcionales han sido explotados, se vuelve poco claro cuales son las correspondientes patas (o más propiamente, las superpatas)  que debemos reemplazar para calcular una superamplitud. \textit{Entonces uno de los objetivos principales de esta tesis, es proveer de manera definitiva, al menos para las teorías más sencillas, formulas para las superpatas en la capa de masa.}

Necesario es explicar porque  no es directo el procedimiento usual (del espacio al superespacio) para obtener dichas patas: Identificar las patas externas con las funciones de onda que provienen de la expansión en modos de Fourier de los campos libres. Un punto medular, que explica en parte porqué los métodos funcionales son más populares en supersimetría, es que estos métodos son naturalmente Lagrangianos. En las versiones Lagrangianas,  la  supersimetría rígida se realiza linealmente en los campos componente, en oposición  a la version Hamiltoniana, que se realiza no linealmente (simetrías no lineales no pueden ser simetrías de la matriz ${S} $). Dicho de otra manera, no es directo que se entiende por un  formalismo Hamiltoniano en el superespacio. Entonces, se vuelve dudoso como generalizar la receta usual en el espacio para obtener las patas en la capa de masa para el caso del superespacio. Esta obstrucción no ha impedido el florecimiento de  modelos supersimétricos realistas. En ninguna extension supersimétrica del modelo estándar, la supersimetría es una  simetría exacta de la matriz $ {S} $~\cite{Dimopoulos:1981zb}. % Pero como hemos argumentado, desde nuestro laboratorio supersimétrico, esta es cuestión  es valida y la búsqueda de su resoluci\'on esta justificada. 


Para abordar el problema de construir patas en la capa de masa, recurrimos a una  formulación  de la teoría cuántica de  los supercampos diferente a los enfoques canónicos y de suma de historias. Extendemos  del espacio al superespacio, lo que hemos  llamado, el enfoque de Weinberg a la teoría cuántica de los campos~\cite{Weinberg:1964cn,Weinberg:1969di,Weinberg:1995mt}. Permeado por la atmósfera de los a\~nos sesenta, donde se respiraba un ``enfoque puro de la matriz $ S $'' a la física de partículas, Weinberg se pregunta ?`Qué suposiciones nos garantizan una dinámica cuántica relativista? ?`Qué consecuencias tienen tales suposiciones?  Weinberg identifica una serie de  principios suficientes para desarrollar una Mecánica Cuántica completamente relativista. Identifica que el operador de Dyson es lo suficientemente general, al menos en el régimen perturbativo, como para garantizar una matriz $ S $ covariante de Lorentz, provisto de que  escribamos el potencial de la teoría  en el esquema de la interacción como una densidad invariante en el espaciotiempo. Reconoce que la utilidad de los operadores de creación-aniquilación radica en que al escribir las interacciones en términos de estos operadores, automáticamente tenemos una dinámica que satisface el principio de  descomposición en cúmulos: La matriz $ S $ nos da probabilidades no correlacionadas para experimentos lo suficientemente separados. Desde el punto de vista puramente epistemológico, Weinberg nos  argumenta del \textit{porqué} de la \textit{inevitabilidad} de los campos cuánticos. Desde una mirada más pragmática y en cierto sentido de más importancia para la física, obtiene las reglas de Feynman para partículas de cualquier espín. Ya que en ningún momento hace uso del formalismo canónico ni del integral de caminos, Weinberg le llama a su enfoque 'no-canónico'. El formalismo de Weinberg es una amalgama de los mundos  puramente de matriz $ S $ y el formalismo Lagrangiano, los campos cuánticos tienen un rol protagónico pero no son construidos a partir de ninguna regla de cuantizaci\'on, sino son los objetos que nos garantizan la invariancia de Lorentz de la teoría cuántica.

El pensar a la teoría cuántica de los supercampos  en estos términos,  nos obliga a repensar a la mecánica cuántica en el superespacio. Un punto en donde creemos haber  ido más lejos de donde se encuentra  la literatura estándar en el tema, es el de haber podido definir el producto interior (o producto escalar) en el espacio de Hilbert para los superestados generales.
Esto nos ha permitido extender una serie de resultados del espacio al superespacio de manera directa. En este trabajo,  damos  un tratamiento unificado para los superestados de partícula masivos y sin masa, introduciendo estados completamente covariantes supersimétricos [usualmente, el análisis del espectro supersimétrico (los llamados supermultipletes) se queda a nivel de estados componente~\cite{julius1992supersymmetry,Weinberg:2000cr}. Aquí presentamos por vez primera a los  estados de superpartículas de manera completamente covariante. 
Explicamos en que sentido, a pesar de que la densidad  Hamiltoniana de interacción no es covariante de super Poincaré, es posible definir una superamplitud completamente covariante supersimétrica. El origen de la no covariancia de la supermatriz $ {S} $ es, desde el punto de vista de este formalismo, el mismo que la no covariancia de Lorentz: La no conmutatividad de la mecánica cuántica, expresada a través de la singularidad de las relaciones de (anti)conmutación de los supercampos, en el ápice del cono de luz. 


En la actualidad no existe un método general para obtener teorías del campo en el superespacio de espín arbitrario y de hecho, solo pocas teorías clásicas del campo  con superespín de dimensión baja se conocen~\cite{Buchbinder:2002gh,Buchbinder:2002tt,Gregoire:2004ic,Gates:2013tka}. Entonces, el formalismo de Weinberg surge como una alternativa a las formulaciones canónicas y/o de integrales de caminos para teorías cuánticas de los supercampos con superespines arbitrarios, independientemente de si estas teorías del campo  pueden ser formuladas en términos canónicos. Este último punto, ha sido enfatizado por Weinberg a  manera de pregunta:

\emph{¿Si descubriéramos una teoría del campo que nos arrojara una matriz $ {S} $ físicamente satisfactoria, nos importaría no poderla derivar de la cuantización canónica de alguna Lagrangiana?}~\cite{Weinberg:1995mt}.

%Aprovechamos que el formalismo de Weinberg nos dice que esperar de cualquier de teoría de los campos en el esquema de la interacción, para buscar directrices y patrones que nos sirvan para discernir si las formulaciones canónicas y/o de la integral de caminos, de donde hipotéticamente se  deriven las superamplitudes obtenidas, son de hecho posibles. Es de nuestra opinión que dichas formulaciones siempre existen,  ya que ciertamente, todas las teorías del campo conocidas aceptan una formulación canónica. 


Probamos el formalismo en la teoría masiva más sencilla, la interacción cúbica del supercampo escalar. Demostramos que la corrección de menor orden al operador de orden temporal, para cuando la superpartícula de superespín cero es su propia antisuperpartícula, corresponde  al potencial de interacción del modelo de Wess-Zumino.  Calculamos la superamplitud de dispersión, en una colisión superpartícula-superantipartícula.  Demostramos, a través de lo compacto y sencillo de las formulas obtenidas, la conveniencia del formalismo propuesto.

Hemos intentando presentar nuestros resultados siguiendo el orden l\'ogico que se presenta en la referencia \cite{Weinberg:1995mt}. En el  capítulo \ref{chap:2}, tratamos todo lo referente a la Mecánica Cuántica en el superespacio, incluyendo las simetrías del superespaciotiempo, mientras que  la introducción de la supermatriz $ S $,  de los operadores de creación-aniquilación y del principio de descomposición en cúmulos   se desarrolla en el capítulo \ref{chap 3}. El ingrediente principal, los supercampos cuánticos son tratados en el capítulo \ref{Chap5}. La presentación de la matriz $ S $ totalmente covariante junto con las correspondientes reglas de super Feynman las damos  en el capítulo \ref{chap:6}.
Todo lo relacionado con las simetrías de  paridad, inversión temporal, conjugación de carga  y simetrías $ \mathcal{R} $,  se presenta en el capítulo \ref{chap:7}. El capítulo \ref{chap:8} esta dedicado  a la aplicación del formalismo para el caso del supercampo escalar. Finalmente, presentamos nuestras conclusiones junto con nuestras perspectivas a futuro en el capítulo \ref{Chap:Conclusiones}. 
 


